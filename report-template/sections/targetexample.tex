\begin{cvletter}
\color{text}
   \subsection{\textsc{Sistema Linux (IP: 192.168.0.0)}}

   \subsubsection{Enumerazione Servizi}


   {{\fontsize{12pt}{1em}\bodyfont\bfseries\textcolor{text}{Nmap Scan Results}}}\\
   \inlinetcb{nmap -sV -sC 10.10.10.10}

   \begin{figure}[H]
      \centering
      \includegraphics[width=1.0\textwidth]{}\\
   \end{figure}

   \begin{center}
      \begin{tabular}{ | l | l | }
       \hline
       \textbf{open TCP ports} & \textbf{open UDP Ports}\\ 
       \hline
       21,22,23 & 100,101\\ 
       \hline
      \end{tabular}
   \end{center}

   Come mostrato dall'immagine sopra, il sistema target ha un webserver sulla porta 80 che\dots

   %{{\fontsize{12pt}{1em}\bodyfont\bfseries\textcolor{text}{Droopescan Scan Results}}}\\
   %{{\fontsize{12pt}{1em}\bodyfont\bfseries\textcolor{text}{Dirb Scan Results}}}\\

   \subsubsection{Vulnerability Explanation}
   \inlinetcb{Ability Server 2.34} e' vulnerabile ad un buffer overflow che causa esecuzione arbitraria di codice\dots

   {{\fontsize{12pt}{1em}\bodyfont\bfseries\textcolor{text}{Severita': \color{awesome-red}{Critica}}}\\

   \subsubsection{Fix Vulnerabilita}
   E' possibile installare una patch per questa vulnerabilita' al seguente \href{http://www.code-crafters.com/abilityserver/}{\underline{Link}}

   \subsubsection{Proof of Concept Code}
   Il sorgente dell'exploit e' reperibile con \inlinetcb{searchsploit -m 10204.py}. Ogni modifica al codice viene 
   evidenziata in rosso. Si ricorda di cambiare opportunamente, dove necessario, gli indirizzi ip e porta rispetto a 
   quelli della propria macchina locale in uso. Preparare, se necessario, un listener netcat con: \inlinetcb{nc -lnvp <PORT>}.

   \begin{tcblisting}{
         %on line,
         %hbox,
         listing only,
         colback=listback,
         colframe=listbordr,
         overlay={
            \begin{tcbclipinterior}
               \fill[listbordr] (frame.south west) rectangle ([xshift=4mm]frame.north west);
            \end{tcbclipinterior}},
         breakable,
         enhanced,
         minted language=python,
         minted options={
            numbersep=2mm,
            linenos,
            highlightlines={2,4,8-9},
            highlightcolor=awesome-red!30
         }
      }
      from os import dunno
      this.foo("fighters")

      while a < 99:
         a += 20

      with open("file.txt") as f:
         readline(f)
         echo 'something'
   \end{tcblisting}


   Dopo aver eseguito l'exploit si dovrebbe ricevere una reverse shell come utente www-data

   \subsubsection{Shell Screenshot}

   \begin{figure}[H]
      \centering
      \includegraphics[width=1.0\textwidth]{}\\
   \end{figure}

   \subsubsection{Elevazione Privilegi}
   C'e' un cronjob lanciato come root\dots

   \subsubsection{Descrizione Vulnerabilita'}

   {{\fontsize{12pt}{1em}\bodyfont\bfseries\textcolor{text}{Severita': \color{awesome-red}{Critica}}}\\



   \begin{figure}[H]
      \centering
      \includegraphics[width=1.0\textwidth]{}\\
   \end{figure}

\end{cvletter}
