\begin{cvletter}

   \begin{minipage}[t]{0.33\textwidth} 

      % Awesome Colors: awesome-emerald, awesome-skyblue, awesome-red, awesome-pink, awesome-orange
      %                 awesome-nephritis, awesome-concrete, awesome-darknight

      \section{\textsc{\color{awesome-red}Ris\color{darktext}ultati}\ \faPieChart \color{gray}\vhrulefill{0.9pt}}}
      \color{text}

      Le vulnerabilità emerse dall'analisi sono classificate con lo standard CVSS v3.1:\\

      \renewcommand{\arraystretch}{2}

      \begin{tabular}{c|ll}
         \large{\textsc{N\textsuperscript{o}}}&\multicolumn{2}{c}{\large{\textsc{Severità (CVSS)}}} \\
         \hline
         \color{black}
         \Huge{\textbf{6}}	     & \large{\textsc{\textbf{\color{black}{Critico}}}} & 9.0--10.0\\
         \hline
         \color{awesome-red}
         \Huge{\textbf{4}}	     & \large{\textsc{\textbf{\color{text}{Alto}}}}& 7.0--8.9\\
         \hline
         \color{awesome-orange}
         \Huge{\textbf{2}}	     & \large{\textsc{\textbf{\color{text}{Medio}}}} & 4.0--6.9\\
         \hline
         \color{awesome-yellow}
         \Huge{\textbf{13}}	     & \large{\textsc{\color{gray}{Basso}}} & 0.1--3.9\\
         \hline
         \color{awesome-skyblue}
         \Huge{123}	     & \normalsize{\textsc{\color{lightgray}{Info}}} & 0.0\\
      \end{tabular}

      %% TORTA
      \begin{tikzpicture}
         \pie[hide number,explode=0.1,text=inside, scale font, radius=2.5, color={black,awesome-red,awesome-orange,awesome-yellow}]{24/\color{white}Critico, 16/\color{white}Alto, 8/\color{white}Medio, 52/\color{white}Basso}
      \end{tikzpicture}

      \footnotesize
      \subsection{Legenda}
      \textsc{\color{black}{\textbf{Critico}}}: Vulnerabilità facilmente sfruttabili che permettono di ottenere accesso completo
      ai sistemi. Si consiglia di prendere provvedimenti immediati.\\

      \textsc{\color{awesome-red}{\textbf{Alto}}}: Vulnerabilità il cui sfruttamento è più difficile ma causano
      perdita di dati o elevazione di privilegi dell'attaccante. Si consiglia di prendere provvedimenti il prima possibile.\\

      \textsc{\color{awesome-orange}{\textbf{Medio}}}: Vulnerabilità che vengono rilevate ma il cui sfruttamento richiede particolari
      condizioni (come social engineering). Tali problemi vanno risolti dopo le categorie di maggior rischio.\\

      \textsc{\color{awesome-yellow}{\textbf{Basso}}}: Vulnerabilità poco dannose, tuttavia la loro
      risoluzione riduce la superficie d'attacco. Prendere provvedimenti nel prossimo ciclo di sviluppo.\\

      \textsc{\color{awesome-skyblue}{\textbf{Info}}}: Categoria che racchiude informazioni aggiuntive raccolte durante i test
      o documentazione.\\
      \sectionsep

      %%%%%%%%%%%%%%%%%%%%%%%%%%%%%%%%%%%%%%
      %
      %     COLUMN TWO
      %
      %%%%%%%%%%%%%%%%%%%%%%%%%%%%%%%%%%%%%%

   \end{minipage} 
   \hfill
   \begin{minipage}[t]{0.60\textwidth} 

      \section{\color{awesome-red}Sin\color{darktext}tesi (\textsc{Executive Summary})\ \color{gray}\vhrulefill{0.9pt}}
      \color{text}
      JamalSec ha eseguito test per la \textbf{valutazione della sicurezza del sito web della Gioielleria Vairo di Pasquale} a
      partire dalla data 1 Dic. 2020 fino al 31 Dic. 2020.\\
      Tramite una serie di attacchi \textbf{sono state trovate molte vulnerabilità critiche} che hanno permesso \textbf{l'accesso
      completo al server remoto con conseguente furto di dati e possibili danni economici}.\\
      Si consiglia di risolvere al più presto tali vulnerabilità in quanto facilmente individuabili da potenziali
      malintenzionati.

      \sectionsep

      \subsection{Riepilogo attacco}

      \begin{enumerate}
            \color{black}
         \item \normalsize E' stata acquisita la password del pannello di amministrazione del webserver tramite ricerca della mail nei data breach pubblici.\\
            \color{lighttext}\small\faExclamationCircle\ Si raccomanda di non usare la mail e username di lavoro per registrarsi in servizi terzi.

            \vspace{0.3cm}

            \color{black}
         \item \normalsize Dal pannello di amministazione e' stato possibile \textbf{acquisire le password degli utenti} salvate nel database. Tali password erano \textbf{direttamente
            leggibili} senza bisogno di effettuare cracking.\\
            \color{lighttext}\small\faExclamationCircle\ Si raccomanda di non salvare in chiaro le credenziali nei database ma usare tecniche di hashing.

            \vspace{0.3cm}

            \color{black}
         \item \normalsize Sfruttando un upload di file non propriamente ristretto all'interno del pannello di amministrazione, e' stato possibile caricare un file nel server che ha permesso un \textbf{accesso con 
            privilegi limitati nel server}.\\
            \color{lighttext}\small\faExclamationCircle\ Si raccomanda di applicare le restrizioni per l'upload di file indicate dalla fondazione owasp.

            \vspace{0.3cm}

            \color{black}
         \item \normalsize \textbf{Una versione datata del sistema operativo} installato nel server (Linux 3.4.5) ha reso possibile la elevazione dei privilegi sfruttando una vulnerabilità nota (CVE--2013--4345) e quindi 
            e' stato ottenuto il controllo completo del server.\\
            \color{lighttext}\small\faExclamationCircle\ Si raccomanda di eseguire periodici aggiornamenti di versione del sistema in uso.
      \end{enumerate}

   \subsection{Altri dettagli}
   Misure sicurezza trovate, da rinforzare\dots
   \end{minipage}


\end{cvletter}
